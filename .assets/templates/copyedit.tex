%!TEX TS-program = xelatex
\documentclass[11pt]{scrartcl}

\usepackage{geometry}
\geometry{verbose,letterpaper,tmargin=2cm,bmargin=2cm,lmargin=2cm,rmargin=2cm}
\usepackage[doublespacing]{setspace}
\usepackage[left]{lineno}

\input{../.assets/templates/pandoc}

\usepackage[hang,flushmargin]{footmisc}

\setlength{\parindent}{0pt}
\setlength{\parskip}{6pt plus 2pt minus 1pt}

\pagestyle{plain}

$if(title)$
\title{$title$}
$endif$
$if(date)$
\date{$date$}
$endif$

% At last:
% The document itself!:

% After filling in all these blanks above, or erasing them
% where they are not needed, Pandoc has finished writing the
% famous LaTeX *preamble* for your document.
% Now comes the all-important command \begin{document}
% which as you can see, will be paired with an \end{document} at the end.
% Pandoc knows whether you have a title, and has already
% specified what it is; if so, it demands that the title be rendered.
% Pandoc knows whether you want a table of contents, you
% specify this on the command line.
% Then, after fiddling with alignments, there comes the real
% business: pandoc slaps its rendering of your text in the place of
% the variable `body`
% It then concludes the document it has been writing.

\begin{document}


$if(title)$
\maketitle
$endif$

{\small
$for(creators)$
\textsc{$creators.givennames$\,$creators.familyname$}\\
$endfor$
}

$if(toc)$
\tableofcontents
$endif$


$if(alignment)$
\begin{$alignment$}
$endif$

\linenumbers

$body$

%$if(alignment)$
\end{$alignment$}
$endif$


\end{document}
